% \section{Controle combinado}
% Conforme vimos na seção \ref{ectq3} podemos controlar um sistema nao linear como  através da técnica do torque computado, usando um controlador PD dado por:
% \begin{equation} \label{ectq3}
% \tau'=\ddot{q}_d+K_v(\dot{q}_d-\dot{q})+K_p(q_d-q) \; ,
% \end{equation}
% sendo $q_{d}$, $\dot{q}_{d}$ e $\ddot{q}_{d}$ a posição desejada, a velocidade desejada e a aceleração desejada; $K_p$
% e $K_v$ são matrizes diagonais $n \times n$, sendo que cada elemento da diagonal é um ganho positivo e escalar.

% Aqui $M_{est}$ e $b_{est}$ são modelos estimados da matriz de inércia, $M$, e do vetor de torques não inerciais, $b$, do robô real,  respectivamente. A equação de malha fechada do sistema é:
% \begin{equation} \label{ectq4}
% \ddot{e}+K_v\dot{e}+K_pe=M_{est}^{-1}[(M-M_{est})\ddot{q}+(b-b_{est})] \; .
% \end{equation}

% Em um manipulador real, podem existir distúrbios externos tais como atrito, variação de torque dos atuadores, e perturbações em virtude  das cargas no robô. Se a soma destes distúrbios for definida como $d_{ext}$ e adicionada à (\ref{ectq4}), teremos
% \begin{equation} \label{ectq5}
% \ddot{e}+K_v\dot{e}+K_pe=M_{est}^{-1}[(M-M_{est})\ddot{q}+(b-b_{est})+d_{ext}] \; .
% \end{equation}

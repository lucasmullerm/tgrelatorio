

\section{Evolução}

O objetivo desta seção é tentar retratar uma evolução da música internacional ao longo dos anos; desde artistas clássicos até os mais modernos. Foram utilizadas 57 composições de um total de 12 artistas e bandas. A composição mais antiga data de 1784 e a mais recente, de 2014. Os artistas utilizados, juntamente com os anos de atividade, são:

\begin{itemize}
    \item Wolfgang Amadeus Mozart (1761 - 1791) \cite{midiworld}
    \item Niccolò Paganini (1793 - 1840) \cite{midimelody}
    \item Frédéric François Chopin (1818 - 1849) \cite{midiworld}
    \item Sergei Vasilievich Rachmaninoff (1877 - 1943) \cite{midiworld}
    \item Scott Joplin (1895 - 1917) \cite{trachtman}
    \item Duke Ellington (1914 - 1974) \cite{midimelody}
    \item Elvis Presley (1953 - 1977) \cite{midiworld}
    \item Antônio Carlos Jobim (1956 - 1994) \cite{wersi}
    \item The Beattles (1960 - 1970) \cite{midiworld}
    \item Pink Floyd (1965 - 1995, 2005, 2012 - 2014) \cite{midiworld}
    \item Dream Theater (1985 - Atualmente) \cite{freemidi}
    \item Taylor Swift (2004 - Atualmente) \cite{freemidi}
\end{itemize}
A figura
% \ref{fig:evo}
mostra o valor médio de entropia de cada composição utilizando como evento o TODO: (nota, cond, dist ou dur). O valor de cada composição está representado com um círculo e o valor médio por artista está representado com um $X$. 

TODO: INSERIR FIGURA

A figura
% \ref{fig:artists}
mostra o valor médio por artista e também apresenta uma escala mais precisa no eixo da entropia.

TODO: INSERIR FIGURA

Nota-se uma pequena difereça entre os compositores mais clássicos (porção mais à esquerda) em relação aos artistas mais modernos, tendo os primeiros valores mais elevados de entropia. Nota-se também que os artistas que 

Para uma comparação mais precisa, as tabelas de
% \ref{tab:mozart} até % \ref{tab:taylor}
mostram os valores de entropia calculados para cada uma das composições em cada um dos modelos já citatos no capítulo \ref{cap:metodo}.
% % Please add the following required packages to your document preamble:
% \usepackage[table,xcdraw]{xcolor}
% If you use beamer only pass "xcolor=table" option, i.e. \documentclass[xcolor=table]{beamer}
\begin{table}[]
\centering
\caption{My caption}
\label{my-label}
\begin{tabular}{|c|c|c|c|c|c|}
\hline
\rowcolor[HTML]{9B9B9B} 
{\color[HTML]{FFFFFF} Composição}                                                                 & {\color[HTML]{FFFFFF} Ano} & {\color[HTML]{FFFFFF} H(Evento)} & {\color[HTML]{FFFFFF} H(Cond)} & {\color[HTML]{FFFFFF} H(Dist)} & {\color[HTML]{FFFFFF} H(Dur)} \\ \hline
\begin{tabular}[c]{@{}c@{}}Piano Sonata\\ No. 11\\ Movement 1\\ Theme and variations\end{tabular} & 1784                       &                                  &                                &                                &                               \\ \hline
\begin{tabular}[c]{@{}c@{}}Piano Sonata\\ No. 11\\ Movement 2\\ Menuetto and trio\end{tabular}    & 1784                       &                                  &                                &                                &                               \\ \hline
\begin{tabular}[c]{@{}c@{}}Piano Sonata\\ No. 11\\ Movement 3\\ Rondo alla turca\end{tabular}     & 1784                       &                                  &                                &                                &                               \\ \hline
Krebsgang                                                                                         & 1785                       &                                  &                                &                                &                               \\ \hline
\begin{tabular}[c]{@{}c@{}}Eine kleine\\ Nachtmusik\end{tabular}                                  & 1787                       &                                  &                                &                                &                               \\ \hline
\end{tabular}
\end{table}
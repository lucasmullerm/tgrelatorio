% Please add the following required packages to your document preamble:
% \usepackage[table,xcdraw]{xcolor}
% If you use beamer only pass "xcolor=table" option, i.e. \documentclass[xcolor=table]{beamer}
\begin{table}[h]
\centering
\caption{Composições de Mozart}
\label{tab:mozart}
\begin{tabular}{|c|c|c|c|c|}
\hline
\rowcolor[HTML]{9B9B9B} 
{\color[HTML]{FFFFFF} Composição}                                                                 & {\color[HTML]{FFFFFF} Ano} & {\color[HTML]{FFFFFF} H(Evento)} & {\color[HTML]{FFFFFF} H(Cond)} & {\color[HTML]{FFFFFF} H(Dist)} \\ \hline
\begin{tabular}[c]{@{}c@{}}Piano Sonata\\ No. 11\\ Movement 1\\ Theme and variations\end{tabular} & 1784                       & 3,07                             & 2,74                           & 2,89                           \\ \hline
\begin{tabular}[c]{@{}c@{}}Piano Sonata\\ No. 11\\ Movement 2\\ Menuetto and trio\end{tabular}    & 1784                       & 3,21                             & 2,96                           & 3,22                           \\ \hline
\begin{tabular}[c]{@{}c@{}}Piano Sonata\\ No. 11\\ Movement 3\\ Rondo alla turca\end{tabular}     & 1784                       & 3,21                             & 2,85                           & 2,74                           \\ \hline
Krebsgang                                                                                         & 1785                       & 2,72                             & 2,71                           & 3,14                           \\ \hline
\begin{tabular}[c]{@{}c@{}}Eine kleine\\ Nachtmusik\end{tabular}                                  & 1787                       & 3,22                             & 2,84                           & 2,56                           \\ \hline
\end{tabular}
\end{table}

% Please add the following required packages to your document preamble:
% \usepackage[table,xcdraw]{xcolor}
% If you use beamer only pass "xcolor=table" option, i.e. \documentclass[xcolor=table]{beamer}
\begin{table}[]
\centering
\caption{Composições de Paganini}
\begin{tabular}{|c|c|c|c|c|}
\hline
\rowcolor[HTML]{9B9B9B} 
{\color[HTML]{FFFFFF} Composição} & {\color[HTML]{FFFFFF} Ano} & {\color[HTML]{FFFFFF} H(Evento)} & {\color[HTML]{FFFFFF} H(Cond)} & {\color[HTML]{FFFFFF} H(Dist)} \\ \hline
Caprice No. 1                     & 1817                       & 3,00                             & 2,62                           & 1,56                           \\ \hline
Caprice No. 2                     & 1817                       & 3,42                             & 3,34                           & 3,83                           \\ \hline
Caprice No. 5                     & 1817                       & 3,31                             & 3,09                           & 3,12                           \\ \hline
Caprice No. 16                    & 1817                       & 3,32                             & 3,11                           & 3,68                           \\ \hline
Caprice No. 24                    & 1817                       & 3,30                             & 3,08                           & 3,68                           \\ \hline
\end{tabular}
\end{table}

% Please add the following required packages to your document preamble:
% \usepackage[table,xcdraw]{xcolor}
% If you use beamer only pass "xcolor=table" option, i.e. \documentclass[xcolor=table]{beamer}
\begin{table}[]
\centering
\caption{Composições de Chopin}
\begin{tabular}{|c|c|c|c|c|}
\hline
\rowcolor[HTML]{9B9B9B} 
{\color[HTML]{FFFFFF} Composição} & {\color[HTML]{FFFFFF} Ano} & {\color[HTML]{FFFFFF} H(Evento)} & {\color[HTML]{FFFFFF} H(Cond)} & {\color[HTML]{FFFFFF} H(Dist)} \\ \hline
Waltzes Op. 70 No. 2              & 1832                       & 3,21                             & 3,10                           & 2,90                           \\ \hline
Nocturnes Op. 37 No. 2            & 1839                       & 3,60                             & 3,52                           & 3,57                           \\ \hline
Fantaisie in F minor Op. 49       & 1841                       & 3,45                             & 3,34                           & 4,57                           \\ \hline
Nocturnes, Op. 48 No. 1           & 1841                       & 3,35                             & 2,9                            & 3,09                           \\ \hline
\end{tabular}
\end{table}

% Please add the following required packages to your document preamble:
% \usepackage[table,xcdraw]{xcolor}
% If you use beamer only pass "xcolor=table" option, i.e. \documentclass[xcolor=table]{beamer}
\begin{table}[]
\centering
\caption{Composições de Scott Joplin}
\begin{tabular}{|c|c|c|c|c|}
\hline
\rowcolor[HTML]{9B9B9B} 
{\color[HTML]{FFFFFF} Composição} & {\color[HTML]{FFFFFF} Ano} & {\color[HTML]{FFFFFF} H(Evento)} & {\color[HTML]{FFFFFF} H(Cond)} & {\color[HTML]{FFFFFF} H(Dist)} \\ \hline
Maple Leaf Rag                    & 1899                       & 3,06                             & 3,00                           & 3,35                           \\ \hline
Peacherine Rag                    & 1901                       & 2,95                             & 2,47                           & 1,44                           \\ \hline
The Easy Winners                  & 1901                       & 3,38                             & 3,06                           & 3,15                           \\ \hline
A Breeze From Alabama             & 1902                       & 3,32                             & 3,13                           & 3,10                           \\ \hline
\end{tabular}
\end{table}

% Please add the following required packages to your document preamble:
% \usepackage[table,xcdraw]{xcolor}
% If you use beamer only pass "xcolor=table" option, i.e. \documentclass[xcolor=table]{beamer}
\begin{table}[]
\centering
\caption{Composições de Duke Ellington}
\begin{tabular}{|c|c|c|c|c|}
\hline
\rowcolor[HTML]{9B9B9B} 
{\color[HTML]{FFFFFF} Composição} & {\color[HTML]{FFFFFF} Ano} & {\color[HTML]{FFFFFF} H(Evento)} & {\color[HTML]{FFFFFF} H(Cond)} & {\color[HTML]{FFFFFF} H(Dist)} \\ \hline
Black Beauty                      & 1928                       & 3,37                             & 3,31                           & 4,30                           \\ \hline
Mood Indigo                       & 1930                       & 3,10                             & 3,29                           & 3,94                           \\ \hline
In A Sentimental Mood             & 1935                       & 3,2                              & 4,13                           & 2,19                           \\ \hline
Take The A Train                  & 1939                       & 3,41                             & 4,43                           & 3,60                           \\ \hline
Satin Doll                        & 1953                       & 2,76                             & 2,26                           & 1,52                           \\ \hline
\end{tabular}
\end{table}

% Please add the following required packages to your document preamble:
% \usepackage[table,xcdraw]{xcolor}
% If you use beamer only pass "xcolor=table" option, i.e. \documentclass[xcolor=table]{beamer}
\begin{table}[]
\centering
\caption{Composições de Elvis Presley}
\begin{tabular}{|c|c|c|c|c|}
\hline
\rowcolor[HTML]{9B9B9B} 
{\color[HTML]{FFFFFF} Composição} & {\color[HTML]{FFFFFF} Ano} & {\color[HTML]{FFFFFF} H(Evento)} & {\color[HTML]{FFFFFF} H(Cond)} & {\color[HTML]{FFFFFF} H(Dist)} \\ \hline
Hound Dog                         & 1956                       & 2,46                             & 2,18                           & 1,58                           \\ \hline
All Shook Up                      & 1957                       & 1,8                              & 1,76                           & 2,18                           \\ \hline
Jailhouse Rock                    & 1957                       & 1,93                             & 1,37                           & 1,21                           \\ \hline
Teddy Bear                        & 1957                       & 2,42                             & 2,19                           & 1,96                           \\ \hline
Are You Lonesome Tonight          & 1969                       & 3,00                             & 3,52                           & 2,62                           \\ \hline
\end{tabular}
\end{table}

% Please add the following required packages to your document preamble:
% \usepackage[table,xcdraw]{xcolor}
% If you use beamer only pass "xcolor=table" option, i.e. \documentclass[xcolor=table]{beamer}
\begin{table}[]
\centering
\caption{Composições de Tom Jobim}
\begin{tabular}{|c|c|c|c|c|}
\hline
\rowcolor[HTML]{9B9B9B} 
{\color[HTML]{FFFFFF} Composição} & {\color[HTML]{FFFFFF} Ano} & {\color[HTML]{FFFFFF} H(Evento)} & {\color[HTML]{FFFFFF} H(Cond)} & {\color[HTML]{FFFFFF} H(Dist)} \\ \hline
Chega de Saudade                  & 1958                       & 3,32                             & 3,65                           & 2,57                           \\ \hline
Desafinado                        & 1959                       & 3,49                             & 3,59                           & 2,91                           \\ \hline
Samba de Uma Nota Só              & 1960                       & 2,28                             & 2,17                           & 1,45                           \\ \hline
Garota de Ipanema                 & 1962                       & 3,16                             & 3,79                           & 2,44                           \\ \hline
Wave                              & 1967                       & 3,24                             & 3,08                           & 2,34                           \\ \hline
\end{tabular}
\end{table}

% Please add the following required packages to your document preamble:
% \usepackage[table,xcdraw]{xcolor}
% If you use beamer only pass "xcolor=table" option, i.e. \documentclass[xcolor=table]{beamer}
\begin{table}[]
\centering
\caption{Composições dos Beattles}
\begin{tabular}{|c|c|c|c|c|}
\hline
\rowcolor[HTML]{9B9B9B} 
{\color[HTML]{FFFFFF} Composição} & {\color[HTML]{FFFFFF} Ano} & {\color[HTML]{FFFFFF} H(Evento)} & {\color[HTML]{FFFFFF} H(Cond)} & {\color[HTML]{FFFFFF} H(Dist)} \\ \hline
I Want to Hold Your Hand          & 1964                       & 2,64                             & 2,16                           & 2,58                           \\ \hline
Come Together                     & 1969                       & 2,83                             & 2,54                           & 2,38                           \\ \hline
Here Comes The Sun                & 1969                       & 3,00                             & 2,71                           & 2,92                           \\ \hline
Hey Jude                          & 1970                       & 3,00                             & 2,89                           & 2,50                           \\ \hline
Let it Be                         & 1970                       & 2,58                             & 2,23                           & 2,63                           \\ \hline
\end{tabular}
\end{table}

% Please add the following required packages to your document preamble:
% \usepackage[table,xcdraw]{xcolor}
% If you use beamer only pass "xcolor=table" option, i.e. \documentclass[xcolor=table]{beamer}
\begin{table}[]
\centering
\caption{Composições de Pink Floyd}
\begin{tabular}{|c|c|c|c|c|}
\hline
\rowcolor[HTML]{9B9B9B} 
{\color[HTML]{FFFFFF} Composição} & {\color[HTML]{FFFFFF} Ano} & {\color[HTML]{FFFFFF} H(Evento)} & {\color[HTML]{FFFFFF} H(Cond)} & {\color[HTML]{FFFFFF} H(Dist)} \\ \hline
Brain Damage                      & 1973                       & 3,28                             & 2,73                           & 1,99                           \\ \hline
Money                             & 1973                       & 2,85                             & 2,97                           & 2,64                           \\ \hline
Wish You Were Here                & 1975                       & 2,77                             & 2,65                           & 2,71                           \\ \hline
Hey You                           & 1979                       & 2,46                             & 2,22                           & 2,11                           \\ \hline
\end{tabular}
\end{table}

% Please add the following required packages to your document preamble:
% \usepackage[table,xcdraw]{xcolor}
% If you use beamer only pass "xcolor=table" option, i.e. \documentclass[xcolor=table]{beamer}
\begin{table}[]
\centering
\caption{Composições de Dream Theater}
\begin{tabular}{|c|c|c|c|c|}
\hline
\rowcolor[HTML]{9B9B9B} 
{\color[HTML]{FFFFFF} Composição} & {\color[HTML]{FFFFFF} Ano} & {\color[HTML]{FFFFFF} H(Evento)} & {\color[HTML]{FFFFFF} H(Cond)} & {\color[HTML]{FFFFFF} H(Dist)} \\ \hline
Pull me Under                     & 1992                       & 2,75                             & 2,41                           & 2,86                           \\ \hline
Overture 1928                     & 1999                       & 3,40                             & 3,64                           & 2,96                           \\ \hline
The Dance of Eternity             & 1999                       & 3,58                             & 3,08                           & 3,20                           \\ \hline
Panic Attack                      & 2005                       & 3,36                             & 3,32                           & 3,13                           \\ \hline
Forsaken                          & 2007                       & 2,61                             & 2,45                           & 2,33                           \\ \hline
\end{tabular}
\end{table}

% Please add the following required packages to your document preamble:
% \usepackage[table,xcdraw]{xcolor}
% If you use beamer only pass "xcolor=table" option, i.e. \documentclass[xcolor=table]{beamer}
\begin{table}[]
\centering
\caption{Composições de Taylor Swift}
\label{tab:taylor}
\begin{tabular}{|c|c|c|c|c|}
\hline
\rowcolor[HTML]{9B9B9B} 
{\color[HTML]{FFFFFF} Composição} & {\color[HTML]{FFFFFF} Ano} & {\color[HTML]{FFFFFF} H(Evento)} & {\color[HTML]{FFFFFF} H(Cond)} & {\color[HTML]{FFFFFF} H(Dist)} \\ \hline
You Belong With Me                & 2008                       & 2,64                             & 2,33                           & 2,19                           \\ \hline
22                                & 2012                       & 2,59                             & 2,24                           & 2,47                           \\ \hline
Knew You Were Trouble             & 2012                       & 2,59                             & 2,24                           & 2,78                           \\ \hline
Blank Space                       & 2014                       & 2,39                             & 1,91                           & 1,29                           \\ \hline
Shake It Off                      & 2014                       & 2,56                             & 1,90                           & 2,48                           \\ \hline
\end{tabular}
\end{table}
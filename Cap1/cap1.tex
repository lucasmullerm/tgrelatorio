A música esteve presente com a humanidade por séculos. Há diversas interpretações e a ela é atribuída diversos significados e intenções. Trata-se de uma manifestação artística e, assim como toda outra, há uma transmissão de informação entre criador, no caso, o compositor; e o consumidor, no caso, o ouvinte.

Além de transmitir informação, as composições musicais apresentem em alto nível, muitas vezes, ordem e padrões beme definidos. Podemos citar características como: estrutura, refrões, repetições, \textit{riffs}, etc.

Informação pode ser mensurada, e diversos estudos tem sido feitos na área há decadas, tendo Claude Shannon, o \"pai da teoria da informação\", desenvolvido um papel muito importante. \cite{shannon}

Com a notação musical utilizada há muitos anos, tem-se uma representação da música bem estruturada e discretizada. Há eventos bem definidos, durações, técnicas, etc. Este trabalho utiliza os conceitos estudados por Shannon e outros para tentar medir a quantidade de informação que uma música transmite.

\section{Motivação}

Diversas aplicações proveniente ou relacionadas a músicas necessitam de uma certa identificação de padrões nesta.

Uma delas é a compressão de arquivos digitais. Assim como textos, composições musicais apresentam padrões que as destacam de outras mídias e formatos. Estas características podem ser aproveitadas quando se quer comprimir arquivos digitais, atividade que se resume a representar padrões de forma simplificada a fim de diminuir a quantidade de bytes necessários para representar uma determinada sequêcia de eventos.

Outra aplicação é a geração automatizada de música. Muito utilizada atualmente em jogos digitais, pode se aproveitar da identificação de padrões em composições existentes.

\section{Objetivo}

Este trabalho tem por objetivo a implementação de uma \textit{framework} para extração de padrões e medição da quantidade de informação transmitida por composições musicais.

\section{Abordagem}

O método que foi utilizado nesse trabalho foi tratar as composições de uma forma similar a sinais digitais para que, de tal forma, seja possível definir os eventos a serem observados e modelá-los como processos estocásticos. Alguns estudos semelhantes foram realizados \cite{artigomit}, onde procurou-se estudar padrões de textos e composições músicais.

\section{Organização do Trabalho}

O trabalho está dividido nos capítulos:

\begin{itemize}
    \item \textbf{Capítulo 1:} introduz o objetivo e abordagem do trabalho;
    \item \textbf{Capítulo 2:} descreve o modelo teórico utilizado;
    \item \textbf{Capítulo 3:} descreve a implementação e expõe resultados obtidos;
    \item \textbf{Capítulo 4:} conclui o trabalho e discute resultados e propostas futuras de continuação;
    \item \textbf{Apêndice A:} pequena amostra de teoria musical que auxilia no entendimento do exposto no trabalho.
\end{itemize}

    
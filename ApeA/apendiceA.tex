\label{ape:notacao}

Esta seção exibe, de forma simplificada, alguns conceitos básicos de teoria e notação musical.

%=============================================
\section{Notas e Notação}

Elemento que designa o som. Cada nota está associada à uma frequência (tom). 
As notas podem ser relacionadas a um alfabeto musical, para viabilizar a representação se uma sequência musical.

A sequência do alfabeto musical é:

Dó ($C$) - Ré ($D$) - Mi ($E$) - Fá ($F$) - Sol ($G$) - Lá ($A$) - Si ($B$) - Dó ($C$)

As letras em parentesis correspondem à notação inglesa.

Esta sequência se repete, e cada uma das repetições é chamada de uma oitava. As notas de oitavas diferentes podem ser representadas seguidas de índices numéricos. Por exemplo, $C_3$ e $C_4$.

Entre as notas mostradas, há ainda as notas chamadas de acidentes, que são identificadas por sustenido ($\sharp$), acima e ($\flat$), abaixo. Exemplo: entre $C$ e $D$ tem-se $C\sharp$ ou $D\flat$. A exceção é que não existe uma nota entre $E$ e $F$ nem entre $B$ e $C$.

A distância entre duas notas musicais adjacentes, por exemplo, $C$ e $C\sharp$ é chamada de um semitom.

\section{Acorde}

Um acorde é um conjunto harmônico de notas. Para propósito desse trabalho, um acorde equivale a mais de uma nota sendo tocada simultaneamente.

\section{Duração}

A cada nota é associada uma duração. Esta duração está relacionada com a duração base da notação que é uma batida.

\section{Compasso}

O compasso é uma forma de dividir a música em pequenos trechos. Existem para marcar a cadência e o ritmo da música.

\section{Escala}

Uma escala musical é uma sequência ordenada de tons, consistindo da manutenção dos intervalos entre as notas.
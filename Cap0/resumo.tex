TODO:

Aqui começa o resumo do referido trabalho. Não tenho a menor idéia do que colocar aqui. Sendo assim, vou inventar. Lá vai: Este trabalho apresenta uma metodologia de controle de posição das juntas passivas de um manipulador subatuado de uma maneira subótima. O termo subatuado se refere ao fato de que nem todas as juntas ou graus de liberdade do sistema são equipados com atuadores, o que ocorre na prática devido a falhas ou como resultado de projeto. As juntas passivas de manipuladores desse tipo são indiretamente controladas pelo movimento das juntas ativas usando as características de acoplamento da dinâmica de manipuladores. A utilização de redundância de atuação das juntas ativas permite a minimização de alguns critérios, como consumo de energia, por exemplo.
Apesar da estrutura cinemática de manipuladores subatuados ser idêntica a do totalmente atuado, em geral suas caraterísticas dinâmicas diferem devido a presença de juntas passivas. Assim, apresentamos a modelagem dinâmica de um manipulador subatuado e o conceito de índice de acoplamento. Este índice é utilizado na sequência de controle ótimo do \mbox{manipulador}.
A hipótese de que o número de juntas ativas seja maior que o número de
passivas  $(n_{a} > n_{p})$  permite o controle ótimo das juntas passivas, uma vez que na etapa
de controle destas há mais entradas (torques nos atuadores das juntas ativas), que
elementos a controlar (posição das juntas passivas). 
A música é uma forma artística consumida por todo o mundo em larga escala; e como toda fonte de entretenimento, é uma forma de se transmitir informação entre o compositor ou artista e o ouvinte. O sistema de notação musical mais conhecido e utilizado atualmente é a partitura e, nesse sistema, uma composição musical é representada de forma similar a um ou mais sinais digitais. Este trabalho apresenta o desenvolvimento de uma \textit{framework} capaz de processar composições musicais com base nos conceitos de Teoria de Informação e identificar características das composições em relação à quantidade de informação por elas transmitida. Os resultados foram obtidos com a análise de diversas composições musicais, desde clássicas até as mais modernas. O estudo feito tem grande relevância no campo de geração de músicas procedurais, isto é, composições músicais geradas por computadores através de algoritmos.
